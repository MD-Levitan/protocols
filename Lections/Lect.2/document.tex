\documentclass{beamer}

\usepackage[T1,T2A]{fontenc}
\usepackage[utf8]{inputenc}

\usepackage{graphicx}
\usepackage{blindtext}
\usepackage{ mathrsfs }
\usepackage[russian,english]{babel}
\usepackage{listings}
\lstset{language=Pascal} 


\author{Деркач Максим Юрьевич}
\title{Криптографические протоколы}
\subtitle{Лекция 2 \\ Атаки на протоколы}
\setbeamercolor{frametitle}{bg=cyan!10}


%\usetheme{lucid}
\begin{document}
	\frame {
		\titlepage
	}

	\frame {
		\frametitle{Ссылки}
		
		\url{http://journals.tsu.ru/pdm2/&journal_page=archive&id=1139&article_id=18544}
		
		\url{https://habr.com/ru/post/475218/}
		
		\url{https://prosecco.gforge.inria.fr/personal/bblanche/proverif/}
	}
	
	\frame{
		\frametitle{Определения}
		\subtitle{Определения}
		\bigskip
		
		\textbf{Определение 1}
		\bigskip
		
		\textbf{Атака} - попытка проведения анализа сообщений протокола и(или) выполнения непредусмотренных протоколом действий в целях нарушения работы протокола и(или) получения информации, составляющей секрет участниов протокола.
		
		\bigskip
		
		Атака успешна, если нарушена безопасность протокола, в том числе:	
		\begin{itemize}
			\item срыв выполнения протокола;
			\item получение секретной информации нарушителем;
			\item нарушение аунтефикации сторон.	
		\end{itemize}
	}

	\frame{
		\frametitle{Определения}
		\bigskip
		\textbf{Определение 2}
		
		\bigskip
		\textbf{Компрометация протокола} - это ситуация, когда протокол не способен достичь тех целей, для которых он предназначен, причем противник получает преимущество только путем манипуляции протоколом.
		\bigskip

	}

		\frame{
		\frametitle{Классификация атак}
		
		\bigskip
		\textbf{Классификация атак по типу направленности}
		
		\bigskip
		\begin{enumerate}
			\item атаки направленные на криптографические алгортмы;
			\item атаки направленные на криптографические методы;
			\item атаки направленные на криптографические протколы.	
		\end{enumerate}
		
	}

		\frame{
		\frametitle{Классификация противников/атакующих}
		
		\bigskip
		\textbf{Противники подразделяются на следующих два класса}
		
		\bigskip
		\begin{itemize}
			\item пассивные противники - они могут перехватывать сообщения, пересылаемые участниками протокола, и анализировать их;
			\item активные противники - они могут делать то же, что и пассивые противники, а также:
				\begin{enumerate}
					\item модифицировать или удалять перехваченные сообщения;
					\item генерировать новые сообщения и посылать их участникам протокола;
					\item выдавать себя за участников протокола.	
				\end{enumerate}	
		\end{itemize}
		
	}

	\frame{
	\frametitle{Основные классы атак на протоколы}	
		\begin{enumerate}
			\item \textbf{Атака посередине (MitM)}: \\
			Класс атак, в котором злоумышленник ретранслирует и изменяет сообщения, проходящие между участниками протокола, причем последние не знают о существование злоумышленника, считая, что общаются непосредственно друг с другом.\\
			Пример: протокол Диффи-Хеллмана \\
			Подтипы: \textbf{Атака подмены (Impersonation)} - попытка подменить одного пользователя другим. Нарушитель, выступая от имени одной из сторон и полностью имитируя ее действия,
			получает в ответ сообщения определенного формата, необходимые для подделки отдельных шагов протокола.
		\end{enumerate}

		\bigskip
	}
	
		\frame{
		\frametitle{Основные классы атак на протоколы}	
		\begin{enumerate}
			\setcounter{enumi}{1}
			\item \textbf{Атака с повторной передачей (Replay Attack)}: \\
			Класс атак, в котором злоумышленник записывает сообщения, проходящие в одном сеансе протокола, а далее повторяет их в новом, выдавая себя за одного из участников нового сеанса.\\
			Пример: Бесключевой протокол Шамира (Lect. 7)
			Подтипы: \textbf{Атака на основе новизны (freshness attack)} - в протоколах передачи ключей данная атака часто применяется для повторного
навязывания уже использованного ранее сеансового ключа;
			
			\textbf{Атака отражением (Reflection Attack) Задержка передачи сообщения (Forced Delay)} - перехват противником сообщения
			и навязывание его в более поздний момент времени.
		\end{enumerate}
		
		\bigskip
	}
	
		\frame{
		\frametitle{Основные классы атак на протоколы}	
		\begin{enumerate}
			\setcounter{enumi}{2}
			\item \textbf{Атака подмены типа (Type Flaw Attack)}: \\
			Класс атак, в котором злоумышленник используя переданные сообщения в легальном сеансе протокола, конструирует новое сообщение и передает его в новом сенасе по видом сообщения другого типа.\\
			Пример: протоколы Wide-Mouth Frog, Деннинга-Сако, Yahalom, Отвея-Рисса (Lect. 7)
			
			
		\end{enumerate}
		
		\bigskip
	}
	
	\frame{
		\frametitle{Основные классы атак на протоколы}	
		\begin{enumerate}
			\setcounter{enumi}{3}
		
			\item \textbf{Комбинированная атака (Interleaving Attack)} - подмена или другой метод обмана, использующий комбинацию данных из ранее выполненных протоколов, в том
числе протоколов, ранее навязанных противником.
			Пример: NSPK (Lect. 5)\\
			Подтипы: \textbf{Атака параллельного сеанса (Parallel Session Attack)} -
			класс атак, в котором злоумышленник инициируеет несколько одновременных сеансов протокола с целью использования сообщений из одного сеанса в другом.
			
		\end{enumerate}
		
		\bigskip
	}
	
		\frame{
		\frametitle{Основные классы атак на протоколы}	
		\begin{enumerate}
			\setcounter{enumi}{4}
			\item \textbf{Атака с известным сеаносвым ключом (Known Key Attack)}:\\
			данная атака заключается в попытке получения информации о долговременном ключе или любой другой ключевой
информации, позволяющей восстанавливать сеансовые ключи для других сеансов протокола.
			
			\item \textbf{Атака с известным разовым ключом (Short Term Secret Attack)}: \\
			Классы атак, в котором злоумышленник получает доступ к временным секретам, используемых в протоколах.
			
		\end{enumerate}
		
		\bigskip
	}
	
		\frame{
		\frametitle{Основные классы атак на протоколы}	
		\begin{enumerate}
			\setcounter{enumi}{6}
			\item \textbf{Атака с неизвестным общим ключом (Unknown Key Share Attack)}: \\
			Класс атак на протоколы с атака, при
которой нарушитель C открывает два сеанса с A и B, выступая в первом случае от
имени B, хотя последний может ничего не знать об этом. При этом в результате будет
сформирован общий ключ между A и B, причем A будет уверен, что сформировал
общий ключ с B, а B будет уверен, что сформировал общий ключ с C. Сам ключ
может быть не известен C.
			Пример: NSPK (Lect. 5)
		\end{enumerate}
		
		\bigskip
	}
	
			\frame{
		\frametitle{Основные классы атак на протоколы}	
		\begin{enumerate}
			\setcounter{enumi}{7}
			\item \textbf{Атака с использованием специально подобранных текстов (Known Key Attack)}:\\
			атака на протоколы типа «запрос — ответ», при которой противник по определенному правилу выбирает
запросы с целью получить информацию о долговременном ключе доказывающего.
		\end{enumerate}
		
		\bigskip
	}

		\frame{
	\frametitle{Основные классы атак на протоколы}	
	\begin{enumerate}
		\setcounter{enumi}{8}
			
		\item \textbf{Атака на основе связывания (Binding Attack)}: \\
		Для криптографических протоколов, построенных на основе асимметричных
		шифрсистем, основной уязвимостью является возможность осуществления подмены
		открытого ключа одного из участников на другой открытый ключ с известной противнику секретной половиной этого ключа. В частности, это позволяет противнику
		узнавать содержание зашифрованных сообщений, отправляемых данному участнику.\\
		
	\end{enumerate}
	
	\bigskip
}
	
	\frame {
		\frametitle{}
		
	}


\end{document}