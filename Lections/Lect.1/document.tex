\documentclass{beamer}


\usepackage[T1,T2A]{fontenc}
\usepackage[utf8]{inputenc}

\usepackage{graphicx}
\usepackage{blindtext}
\usepackage{ mathrsfs }
\usepackage[russian,english]{babel}

\author{Деркач Максим Юрьевич}
\title{Криптографические протоколы}
\subtitle{Лекция 1}

%\usebackgroundtemplate{
%	\includegraphics[width=\paperwidth,height=\paperheight]{../bsu.jpg}
%}

\setbeamercolor{frametitle}{bg=cyan!10}
\usepackage{color}


%\usetheme{lucid}
\begin{document}
	
	\frame {
		\titlepage
	}
	\frame{
		\frametitle{Содержание учебного материала}
		%\framesubtitle{An Example of Lists}
		\begin{itemize}
			\item Основные понятия.
			\item Атаки на протоколы.
			\item Управление ключами, их классификация и жизненный цикл.
			\item Протоколы аутентификации.
			\item Протоколы распределения и обновления ключей.
			\item Протоколы электронного голосования.
			\item Защищенные каналы передачи данных (IPSEC, SSL/TLS, ...).
		\end{itemize}
	}
	\frame{
		\frametitle{Литература}
		\begin{itemize}
			\item А.В. Соколов "Защита информации в распределенных корпаративных сетях и системах"
			\item А.М. Миронов "Криптографические протоколы"
		\end{itemize}
	}
	
	\frame{
		\frametitle{Список основных обозначений}
		\begin{itemize}
			\item $A, B$ - участники инфор. обмена
			\item $S$ - центр распределения ключей (3-ая доверенная сторона)
			\item $I$ - злоумышленник
			\item $ID_x$ - идентификатор $X$
			\item $K_{xy}$ - общий секретный ключ $X,Y$
			\item $KS$ - секретный сеансовый ключ
			\item $K_x^{pub}$ - открытый ключ $X$
			\item $K_x^{sec}$ - секретный ключ $X$
			\item $N_x$ - порядковый номер $X$
			\item $R_x, Nonce_x$ - (number used once) случайное число, выработанное $X$
		
		\end{itemize}
		
	}
	\frame{
	\frametitle{Список основных обозначений}
	\begin{itemize}
		\item $T_X$ - временная отметка, поставленная $X$
		\item $TVP_x$ - одноразовый параметр $X$
		\item $T_x/N_x$ - одноразовый параметр $X$, который является либо временной меткой, либо порядковым номером
		\item $E_k(M)$ - шифрование на ключе $k$
		\item $D_k(M)$ - расшифрование на ключе $k$
		\item $MAC_k(M)$ - выработка имитовставки.
		\item $h, h(M)$ - выработка хэша
		\item $Sign_{K_x^{sec}}(M)$ - ЭЦП сообщения $M$ участника $X$
		\item $Cert_x$ - сертификат участника $X$
		\item $M_1 || M_2$ - конкатенация
	\end{itemize}	
	}
	\frame {
	\frametitle{Определения}
	
	\textbf{Определение 1}
	
	\bigskip
	
	 	 \textbf{Протокол} - совокупность действий выполняемых в заданной последовательности двумя или более сторонами с целью достижения определенного результата.
	 
	 \bigskip
	 \bigskip
	 
	 \textbf{Определение 2}
	 
	 \bigskip
	 
	  \textbf{Криптографический протокол} - протокол, в котором используются криптографические средства (алгоритмы).
	 
	 \bigskip
	 \bigskip
	 
	 \textbf{Определение 3}
	 
	 \bigskip
	 
	  \textbf{Сеанс} - это однократное выполнение протокола.
	}

	\frame {
	\frametitle{Свойства протокола}
		\begin{enumerate}
			\item Действия имеют строгую очередность от начала и до конца (ни одно действие не выполняется, пока не закончится другое).
			
			\item Должно быть точно определено каждое действие.
			
			\item Все стороны, участвующие в протоколе, должны заранее знать последовательность действий.
		\end{enumerate}
	}

		\frame {
		\frametitle{Классификация криптографических протоколов}
		
		\begin{enumerate}
			\item на основе задач
			
			\item по числу  участников в протоколе
			
			\item по числу передаваемых сообщений
		\end{enumerate}
	}

	\frame {
		\frametitle{Модель угрозы Долева-Яо}
		\framesubtitle{(Dolev-Yao)}
		Возможности злоумышленника:
		\begin{enumerate}
			\item[+] перехватывать $\forall$ сообщение в сети
			\item[+] вступать в контакт с другим пользователем
			\item[+] получать сообщение от $\forall$ пользователя
			\item[+] посылать сообщение $\forall$ пользователю, маскируясь под $\forall$ другого пользователя
		\end{enumerate}
		
		Злоумышленник не может:
		
		
		\begin{enumerate}	
			\item[-] угадать случайное число, выбранное из достаточно большого множества
			\item[-] восстановить открытый ключ по шифротексту, не имея правильного секретного ключа
			\item[-] зашифровать исходное сообщение
			\item[-] найти  личный ключ, имея соответствующий открытый ключ
			\item[-] иметь доступ к закрытым зонам вычислительной среды
		\end{enumerate}
	}
	\frame {
	\begin{figure}
		%\includegraphics[width=0.8\linewidth]{index1.jpeg}
		
	\end{figure}
	}

\end{document}