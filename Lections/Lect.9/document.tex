\documentclass{beamer}




\usepackage[T1,T2A]{fontenc}
\usepackage[utf8]{inputenc}

\usepackage{graphicx}
\usepackage{blindtext}
\usepackage{ mathrsfs }
\usepackage[russian,english]{babel}
%\usepackage{ dsfont }

\author{Деркач Максим Юрьевич}
\title{Криптографические протоколы}
\subtitle{Лекция 9 \\ Протоколы распределения ключей (Часть 3)}


%\usetheme{lucid}
\begin{document}
	\frame {
		\titlepage
	}
	\frame {
		\frametitle{Ссылки}
		
		\begin{enumerate}
			\item ISO/IEC 11770-1:2010 – Information technology – Security techniques – Key management – Part 1: Framework 
			\item ISO/IEC 11770-2:2008 – Information technology – Security techniques – Key management – Part 2: Mechanisms using symmetric techniques 
			\item ISO/IEC 11770-3:2008 – Information technology – Security techniques – Key management – Part 3: Mechanisms using asymmetric techniques 
			\item ISO/IEC 11770-4:2006 – Information technology – Security techniques – Key management – Part 4: Mechanisms based on weak secrets
			
			\item СТБ 34.101.45-2013 "Информационные технологии и безопасность. Алгоритмы электронной цифровой подписи и транспорта ключа на основе эллиптических кривых". \hyperlink{name}{http://apmi.bsu.by/assets/files/std/bign-spec19.pdf} 
			
			\item СТБ 34.101.60-2014 "Информационные технологии и безопасность. Алгоритмы разделения секрета". \hyperlink{name}{http://apmi.bsu.by/assets/files/std/bels-spec12.pdf} 
		\end{enumerate}
	}
	
	\frame{
		\frametitle{Протоколы распределения ключей}
		\framesubtitle{Предварительное распределение ключей}
		
		Предварительное распределение ключей нужно для уменьшения объёма распределяемой и хранимой информации. \\
		\bigskip
		$A_1, ..., A_n$ - абоненты.\\
		$K$ - множество ключей. \\
		$P$ - множество исходных ключевых параметров ($p_i$ - пароль каждого абонента).\\
		$Q$ - множество значений ключевых материалов абонентов ($q_i$ - секрет каждого абонента). \\
		$R$ - множество значений открытой информации ($r_1, ..., r_n$ - в открытом доступе). \\
		
		
	}
	
	\frame{
		\frametitle{Протоколы распределения ключей}
		\framesubtitle{Предварительное распределение ключей}
		
		Схема предварительного распределения ключей:\\
		$S(n) = (K,P, Q,R, A_0, A_1)$ \\
		
		\begin{enumerate}
			\item $A_0 : P \times R -> Q$ - алгоритм формирования секретных ключевых материалов.\\
			$A_0(p_i, r_i) = q_i,\ 1 \leq i \leq n$
			\item $A_1 : Q \times R -> K$ - алгоритм вычисления ключа парной связи.\\
			$A_1(q_i, r_j) = A_1(q_j, r_i),$\\
			$K_{ij} = A_1(q_i, r_j), i=j : $ либо не рассматривается либо некий личный секретный ключ.\\
			
		\end{enumerate}
		$A_0(p, r_i) = Q_i \subseteq K^{t_i} \subseteq Q, \ 1 \leq i \leq n$
	}
	
	\frame{
		\frametitle{Протоколы распределения ключей}
		\framesubtitle{Предварительное распределение ключей}
		
		\textbf{Предложение 1}\\
		$\forall r_i \in R,\ q_i \in Q,\ 1 \leq i \leq n$\\
		$A_0(p, r_i) = q_i $ - имеет одинаковое число решений относительно $p \in P$.
		
		\bigskip
		
		\textbf{Предложение 2}\\
		$\forall r_i \in R,\ k \in K,\ 1 \leq i \leq n$\\
		$A_1(q_i, r_i) = k $ - имеет одинаковое число решений относительно $q_i \in Q$.		
	}

	\frame{
		\frametitle{Протоколы распределения ключей}
		\framesubtitle{Предварительное распределение ключей}
		
		
		Пусть $1 \leq m \leq n -2 \ \ S(n)$ является стойкой к  $m$-кратной компрометации ключей 
		(к сговору $m$ абонентов), если после того как злоумышленнкиу станут известны 
		ключевые материалы $m$ абонентов ($q_1 ,..., q_m$ - не ограничивая общности) 
		он не сможет получить никакой информации о ключах парной связи остальных абонентов
		($K_{i, 1}, ..., K_{i, m} \ \forall A_i,\ m + 1 \leq i \leq n$).
		
		\bigskip
		
		\textbf{Предложение 3}\\
		$\forall r_1, ..., r_{m + 1} \in R,$ \\
		$K_{i, 1}, ..., K_{i, m+1} \in K$, \\
		то система имеет одинаковое число решений относительно $q_i \in Q$.
		
		\bigskip
		
		\textbf{Теорема}\\
		$S(n)$, удолетворяющая предположению 1, является стойкой к $m$-кратной 
		компрометации ключей $\leftrightarrow$ когда выполнено предположение 3.
		
	}	

	\frame{
	\frametitle{Протоколы распределения ключей}
	\framesubtitle{Предварительное распределение ключей}
	
	\textbf{Следствие}\\
	Если $S(n)$ является стойкой к $m$-кратной компрометцаии, то
	\begin{enumerate}
		\item каждый абонент должен хранить не менее $(m + 1)log_2{(K)}$
			  бит ключевых материалов;
		\item центр распределения ключей должен хранить не менее $\frac{(n + 1)n}{2}log_2{(K)}$
			  бит исходных ключевых материалов.
	\end{enumerate}
	
	\bigskip
	
	Схема $S(n)$ называется оптимальной, если для неё выполняются нижние границы указанных выше ограничений.
	
	}
	
	\frame{
		\frametitle{Протоколы распределения ключей}
		\framesubtitle{Предварительное распределение ключей}
		
		\textbf{Схема Блома}\\
		$F$ - конечное поле, имеющее достаточно большое число элементов($n$ эелементов).\\
		$r_1, ..., r_n\ \in F \ne 0 \ r_i \rightarrow A_i$\\
		$f(x,y) = \sum_{s=0}^{m} \sum_{t=0}^{m} a_{st}x^sy^t,\ a_{st}=a_{ts},\ s \ne t, \ s,t = 0, ..., m$ \\
		$1 \leq m \leq n - 2,\ a_{st}$ - секретные материалы, хранимые только в центре распределения. \\
		$A_i :\ q_i=(a_0^{(i)}, a_1^{(i)}, ..., a_m^{(i)})$\\
		$q_i(x) = f(x, r_i) = a_0^{(i)} + a_1^{(i)}x + ... + a_m^{(i)}x^m$\\
		$K_{ij} = K_{ji} = f(r_i, r_j) = q_i(r_j) = q_j(r_i)$ \\
		$A_i$ хранит $m+1$ значение ключевых паролей. \\
		Схема Блома является стойкой к  $m$-кратной компрометации ключей.
				
	}
	
	\frame {
		\frametitle{}
		
	}
	
	
	\frame {
		\begin{figure}
			\includegraphics[width=0.8\linewidth]{index1.jpeg}
			
		\end{figure}
	}
	
\end{document}