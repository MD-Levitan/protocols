\documentclass[a4paper,12pt]{article}
\usepackage[T1,T2A]{fontenc}
\usepackage[utf8]{inputenc}


\usepackage{graphicx}
\usepackage{blindtext}
\usepackage{ mathrsfs }
\usepackage[russian,english]{babel}
\usepackage{enumitem}
\usepackage[english,russian]{babel}
\usepackage{amsmath}
\usepackage[top=20mm, bottom=20mm, left=20mm, right=20mm]{geometry}
\usepackage[final]{pdfpages}

\author{Деркач Максим Юрьевич}

\begin{document}
	
	\section{Вопрос 1}
	Криптографические протоколы: основные понятия.Модель угроз Долева-Яо.
	\begin{enumerate}
		\item Основные определения.
		\item Свойства криптографических протоколов.
		\item Классификация криптографических  протоколов.
		\item Модель Долева-Яою
	\end{enumerate}
	\newpage
	
	\section{Вопрос 2}
	Атаки на криптографические протоколы.
	\begin{enumerate}
		\item Основные определения.
		\item Классификация атак на криптографические  протоколы.
		\item Дать определение, перечислить подтипы и привести пример для следующих классов атак:
		\begin{enumerate}
			\item Атака посередине
			\item Атака с повторной передачей
			\item Атака подмены типа
			\item Комбинированная атака
			\item Атака с известным сеансовым ключом
			\item Атака с неизвестным общим ключом
			\item Атака с использованием специально подобранных текстов
			\item Атака на основе связывания
			
		\end{enumerate}    
	\end{enumerate}
	\newpage
	
	\section{Вопрос 3}
	Управление ключами. Классификация ключей. Жизненный цикл ключей. Особенности управления ключами в симметричных и асимметричных криптосистемах
	\begin{enumerate}
		\item Основные определения
		\item Классификация ключей
		\item Жизненный цикл ключей
		\item Особенности управления ключами
	\end{enumerate}
	\newpage
	
	\section{Вопрос 4}
	Протоколы аутентификации: классификация, атаки. Протоколы слабой аутентификации
	\begin{enumerate}
		\item Основные определения
		\item Классификация аутентификации
		\item Фиксированные пароли, HTTP authentication
		\item Одноразовые пароли, Схема Лэмпорта
	\end{enumerate}
	\newpage
	
	
	\section{Вопрос 5}
	Протоколы сильной аутентификации
	\begin{enumerate}
		\item Основные определения
		\item Классификация протоколов сильной аутентификации
		\item Напишите протоколы ISO/ IEC 9798 - 2
		\item Напишите протокол Ву-Лама и атаку к данному протоколу
		\item Напишите протокол NSPK и атаку к данному протоколу
		\item Напишите протокол сильной аутентификации с использованием ЭЦП
	\end{enumerate}
	\newpage
	
	\section{Вопрос 6}
	Протоколы аутентификации на основе техники доказательства знания
	\begin{enumerate}
		\item Основные определения
		\item Схема протокола
		\item Напишите протокол Фиата-Шамира
		\item Напишите протокол Шнора
		\item Напишите протокол GQ
	\end{enumerate}
	\newpage
	
	\section{Вопрос 7}
	Протоколы распределения ключей на основе симметричных криптосистем
	\begin{enumerate}
		\item Основные определения
		\item Классификация протоколов распределения ключей
		\item Напишите протоколы ISO/IEC 11770-2
		\item Определение "Коммутирующее шифрующее преобразование", Трёхпроходный протокол Шамира
		\item Напишите протокол Wide-Mouth-Frog и атаку к данному протоколу
		\item Напишите протокол Needham-Schroeder (NSSK) и атаку к данному протоколу
		\item Напишите протокол Отвея - Рисса и атаку к данному протоколу
	\end{enumerate}
	\newpage
	
	\section{Вопрос 8}
	Протоколы распределения ключей на основе aсимметричных криптосистем
	\begin{enumerate}
		\item Основные определения
		\item Классификация протоколов распределения ключей
		\item Напишите протокол Needham-Schroeder Public Key (NSPK)
		\item Напишите протокол NSPK без 3-ей стороны
		\item Напишите протокол EKE(Encrypted Key Exchange)
		\item Напишите протокол распределения ключей с использованием ЭЦП
		\item Напишите протокол MTI
	\end{enumerate}
	\newpage
	
	\section{Вопрос 9}
	Предварительное распределение ключей. Протоколы голосования
	\begin{enumerate}
		\item Основные определения
		\item Классификация протоколов распределения ключей
		\item Напишите cхему Шамира
		\item Напишите cхему разделения секрета Блома
		\item Напишите протоколы голосования с ЦИК и ЦУР
		\item Напишите улучшенный протокол голосования
		\item Опишите гомоморфное шифрование в протоколах голосования
	\end{enumerate}
	\newpage
	
	\section{Вопрос 10}
	Протоколы SSL/TLS, СТБ 34.101.65
	\begin{enumerate}
		\item Описание протокола(ов): функционал, версии
		\item Описание шагов протокола
		\item Алгоритмы  формирования  общего  ключа
		\item Методы  аутентификации
	\end{enumerate}
	\newpage
	
	\section{Вопрос 11}
	Протокол IPSEC
	\begin{enumerate}
		\item Описание протокола(ов): функционал, версии
		\item Описание шагов протокола
		\item Архитектура протокола
		\item Security Association
	\end{enumerate}
	\newpage
	
	\section{Вопрос 12}
	Протоколы ESP, AH
	\begin{enumerate}
		\item Описание протокола(ов): функционал, версии
		\item Описание структуры пакетов
	\end{enumerate}
	\newpage
	
	\section{Вопрос 13}
	Протокол SSH
	\begin{enumerate}
		\item Описание протокола(ов): функционал, версии
		\item Описание шагов протокола
		\item Архитектура протокола
		\item SSH-TRANS
		\item SSH-USERAUTH
	\end{enumerate}
	\newpage
	
	\section{Вопрос 14}
	Сравнение протоколов SSH, IPSEC, SSL. AEAD-режим шифрования
	\begin{enumerate}
		\item Сравнение протоколов SSH, IPSEC, SSL
		\item Описание AE-, AEAD-режимов
		\item Определение "Неразличимость шифротекста"
		\item Определение "Неизменяемость шифротекста"
		\item Определение "Целостность открытого текста"
	\end{enumerate}
	\newpage
	
	\section{Вопрос 15}
	Протокол WEP, атаки
	\begin{enumerate}
		\item Описание протокола(ов): функционал, версии
		\item Описание шагов протокола
		\item Архитектура протокола
		\item Атаки на протокол
	\end{enumerate}
	\newpage
	
	\section{Вопрос 16}
	Протоколы WPA, WPA2, WPA3 и атаки
	\begin{enumerate}
		\item Описание протокола(ов): функционал, версии
		\item Различие WEP и WPA, WPA и WPA2
		\item Описание шагов протокола
		\item Архитектура протокола
		\item Атаки на протокол
	\end{enumerate}
	\newpage
	
	\section{Вопрос 17}
	СТБ 34.101.66-2014 "Информационные технологии и безопасность. Протоколы формирования общего ключа на основе эллиптических кривых".
	\begin{enumerate}
		\item Описание протокола(ов): функционал, версии
		\item Шаги протокола и входные/выходные данные (один из BMQV, BSTS, BPACE)
	\end{enumerate}
	\newpage
	
\end{document}