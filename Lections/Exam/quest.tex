\documentclass[a4paper,12pt]{article}
\usepackage[T1,T2A]{fontenc}
\usepackage[utf8]{inputenc}


\usepackage{graphicx}
\usepackage{blindtext}
\usepackage{ mathrsfs }
\usepackage[russian,english]{babel}
\usepackage{enumitem}
\usepackage[english,russian]{babel}
\usepackage{amsmath}
\usepackage[top=20mm, bottom=20mm, left=20mm, right=20mm]{geometry}
\usepackage[final]{pdfpages}

\author{Деркач Максим Юрьевич}

\begin{document}
	
	\section{Вопрос 1}
	Криптографические протоколы: основные понятия.Модель угроз Долева-Яо.
	\begin{enumerate}
		\item Основные определения.
		\item Свойства криптографических протоколов.
		\item Классификация криптографических  протоколов.
		\item Модель Долева-Яою
	\end{enumerate}
	\newpage
	
	\section{Вопрос 2}
	Атаки на криптографические протоколы.
	\begin{enumerate}
		\item Основные определения.
			\item Классификация атак на криптографические  протоколы.
		\item Дать определение, перечислить подтипы и привести пример для следующих классов атак:
		\begin{enumerate}
			\item Атака по середине
			\item Атака с повторной передачей
			\item Атака подмены типа
			\item Комбинированная атака
			\item Атака с известным сеаносвым ключом
			\item Атака с неизвестным общим ключом
			\item Атака с использованием специально подобранных текстов
			\item Атака на основе связывания
			
		\end{enumerate}	
	\end{enumerate}
	\newpage
	
	\section{Вопрос 3}
	Управление ключами \\ Жизненый цикл ключей
	\begin{enumerate}
		\item Основные определения
		\item Классификация ключей
		\item Жизненный цикл ключей
		\item Особенности управления ключами
	\end{enumerate}
	\newpage
	
	\section{Вопрос 4}
	Протоколы аутентификации: классификация, атаки. Протоколы "слабой" аутентификации
	\begin{enumerate}
		\item Основные определения
		\item Классификация аутентификации
		\item Фиксированные пароли, HTTP authentication
		\item Одноразовые пароли, Схема Лэмпорта
	\end{enumerate}
	\newpage
	
	
	\section{Вопрос 5}
	Протоколы сильной аутентификации
	\begin{enumerate}
		\item Основные определения
		\item Классификация протоколов сильной аутентификации
		\item Напишите протоколы ISO/ IEC 9798 - 2
		\item Напишите протокол Ву-Лама и атаку к данному протоколу
		\item Напишите протокол NSPK и атаку к данному протоколу
		\item Напишите протокол сильной аутентификации с использованием ЭЦП
	\end{enumerate}
	\newpage
	
	\section{Вопрос 6}
	Протоколы на основе техники доказательства знания
	\begin{enumerate}
		\item Основные определения
		\item Схема протокола
		\item Напишите протокол Фиата-Шамира
		\item Напишите протокол Шнора
		\item Напишите протокол GQ
	\end{enumerate}
	\newpage
	
	\section{Вопрос 7}
	Протоколы распределения ключей на основе симметричной кс
	\begin{enumerate}
		\item Основные определения
		\item Классификация протоколов распределения ключей
		\item Напишите протоколы ISO/IEC 11770-2
		\item Определение "Коммутирующее шифрующее преобразование", Трёхпроходный протокол Шамира
		\item Напишите протокол Wide-Mouth-Frog и атаку к данному протоколу
		\item Напишите протокол Needham-Schroeder (NSSK) и атаку к данному протоколу
		\item Напишите протокол Отвея - Рисса и атаку к данному протоколу
	\end{enumerate}
	\newpage

	
	
\end{document}