\documentclass{beamer}




\usepackage[T1,T2A]{fontenc}
\usepackage[utf8]{inputenc}

\usepackage{graphicx}
\usepackage{blindtext}
\usepackage{ mathrsfs }
\usepackage[russian,english]{babel}
%\usepackage{ dsfont }

\author{Деркач Максим Юрьевич}
\title{Криптографические протоколы}
\subtitle{Лекция 8 \\ Протоколы распределения ключей (Часть 2)}


%\usetheme{lucid}
\begin{document}
	\frame {
		\titlepage
	}
	\frame {
		\frametitle{Ссылки}
		
		\begin{enumerate}
			\item ISO/IEC 11770-1:2010 – Information technology – Security techniques – Key management – Part 1: Framework 
			\item ISO/IEC 11770-2:2008 – Information technology – Security techniques – Key management – Part 2: Mechanisms using symmetric techniques 
			\item ISO/IEC 11770-3:2008 – Information technology – Security techniques – Key management – Part 3: Mechanisms using asymmetric techniques 
			\item ISO/IEC 11770-4:2006 – Information technology – Security techniques – Key management – Part 4: Mechanisms based on weak secrets
			
			\item СТБ 34.101.45-2013 "Информационные технологии и безопасность. Алгоритмы электронной цифровой подписи и транспорта ключа на основе эллиптических кривых". \hyperlink{name}{http://apmi.bsu.by/assets/files/std/bign-spec19.pdf} 
			
			\item СТБ 34.101.60-2014 "Информационные технологии и безопасность. Алгоритмы разделения секрета". \hyperlink{name}{http://apmi.bsu.by/assets/files/std/bels-spec12.pdf} 
		\end{enumerate}
	}
	
	\frame{
		\frametitle{Протоколы распределения ключей}
		\framesubtitle{Needham-Schroeder (NSSK)}
		
		\begin{enumerate}
			\item $A -> S: ID_A || ID_B || R_A$
			\item $S -> A: E_{K_{AS}}(R_A || ID_B || KS || E_{K_{BS}}(KS || ID_A))$
			
			\item $A -> B: E_{K_{BS}}(KS || ID_A)$
			\item $B -> A: E_{KS}(R_B)$
			\item $A -> B: E_{KS}(R_B - 1)$			   
		\end{enumerate}
	}

		\frame{
		\frametitle{Протоколы распределения ключей}
		\framesubtitle{Needham-Schroeder (NSSK)}
		
		
		\textbf{Атака} \\
		
		Если ключ $KS$	скомпрометирован, возможна атака на протокол методом повтора сеанса: берутся	сообщения из прошлого сеанса с ключом $KS^{*}$: 
		
		\begin{enumerate}
			\item $A -> S: ID_A || ID_B || R_A$
			\item $S -> A: E_{K_{AS}}(R_A || ID_B || KS || E_{K_{BS}}(KS || ID_A))$
			
			\item $I(A) -> B: E_{K_{BS}}(KS^{*} || ID_A)$
			\item $B -> I(A): E_{KS^{*}}(R_B)$
			\item $I(A) -> B: E_{KS^{*}}(R_B - 1)$			   
		\end{enumerate}
		
	}
	
	\frame{
		\frametitle{Протоколы распределения ключей}
		\framesubtitle{KERBEROS}
	
		Несколько видоизмененный протокол   Needham  –  Schroeder  был положен в основу программного средства аутентификации пользователей распределенных вычислительных систем Kerberos.
		\\		
		В целях исключения возможности осуществления атаки, описанной выше, клиент,  пройдя аутентификацию на сервере аутентификации, должен предварительно, до того, как ему будет предоставлен доступ к серверам приложений, получить у специального сервера выдачи билетов так называемые билеты  – структуры данных,  в которых указывается срок полномочий клиента для доступа к серверам приложений. По истечении этого срока клиент должен получать новый билет. Эта мера ограничивает срок, в течение которого возможно осуществить атаку на протокол.
		
	}
	
	\frame{
		\frametitle{Протоколы распределения ключей}
		\framesubtitle{KERBEROS}
		
		$AS$ - сервер аутентификации\\
		$TGS$ - сервер выдачи билетов\\
		$ticket_1 = ID_{TGS} || E_{K_{AS,TGS}}(ID_A|| ID_{TGS} || T_{AS} || L || K_{A,TGS})$ - \\
		$auth_1 = E_{K_{A,TGS}}(ID_A || T_A || ...)$\\
		$ticket_2 = ID_B || E_{K_B}(ID_A || ID_B || ID_{TGS}|| L^{'} || K)$\\
		$auth_2 = E_{K}(ID_A || T_A || K_A)$\\
		$auth_3 = E_{K}(ID_A || T_A + 1 || K_B)$\\
		
				
	}

	\frame{
		\frametitle{Протоколы распределения ключей}
		\framesubtitle{KERBEROS}
	
		\begin{enumerate}
		\item $A -> AS: \ \ ID_A || ID_{TGS} || R_A$
		\item $AS -> A: \ \ E_{K_{A, AS}}( E_{K_{A,TGS}} || R_A || ticket_1)$
		\item $A -> TGS: ID_B || ticket_1 || auth_1$
		\item $TGT -> A: E_{K_{A,TGS}}(K || ticket_2)$	
		\item $A -> B: \ \ \ \ ticket_2 || auth_2$
		\item $B -> A: \ \ \ \ auth_3$	
		\end{enumerate}	
	
		Шаги (1) – (2) выполняются только во время первого входа клиента в систему.\\
		Шаги (3) – (4) выполняются всякий раз ,  когда клиент $A$ хочет обратиться к новому серверу $B$.  \\	
		Шаг (5) выполняется всякий раз, когда $A$ проходит аутентификацию для $B$. \\	
	 	Шаг (6) является необязательным и выполняется,  когда $A$ требует от $B$ взаимную аутентификацию.
	 	
	}
	
	
	\frame{
		\frametitle{Протоколы распределения ключей}
		\framesubtitle{Протоколы основанные на ассиметричных криптосистемах}
		
		\textbf{Needham-Schroeder Public Key (NSPK)}

		\begin{enumerate}
			\item $A -> S: ID_A || ID_B $
			\item $S -> A: E_{K_{S}^{sec}}(K_B^{pub} || ID_B)$
	
			\item $A -> B: E_{K_{B}^{pub}}(K_A || ID_A)$
			\item $B -> S: ID_B || ID_A$
			\item $S -> B: E_{K_{S}^{sec}}(K_A^{pub} || ID_A)$
			
			\item $B -> A: E_{K_{A}^{pub}}(K_B || K_A)$
			\item $A -> B: E_{K_{B}^{pub}}(K_B)$
			\item $A,B:	\ \ \ \ \ KS = f(K_A, K_B)$   
		\end{enumerate}
		
		$E_{K_{S}^{sec}}()$ - подпись на секретном ключе.\\
		$E_{K_{B}^{pub}}()$ - шифрование на открытом ключе.\\
		$f()$ - общеизвестная однонаправленная функция. 
		
	}

		\frame{
		\frametitle{Протоколы распределения ключей}
		\framesubtitle{Протоколы основанные на ассиметричных криптосистемах}
		
		\textbf{NSPK без 3-ей стороны}
		
		\begin{enumerate}
			\item $A -> B: E_{K_{B}^{pub}}(K_A || ID_A)$
			\item $B -> A: E_{K_{A}^{pub}}(K_A || K_B)$
			\item $A -> B: E_{K_{B}^{pub}}(K_B)$
			\item $A,B:	\ \ \ \ \ KS = f(K_A, K_B)$   
		\end{enumerate}
		
	}
	
	\frame{
		\frametitle{Протоколы распределения ключей}
		\framesubtitle{Смешанные протоколы}
		
		\textbf{EKE(Encrypted Key Exchange)}\\
		$K_{AB} = P $ - пароль
		\begin{enumerate}
			\item $A -> B: ID_A || E_P(K_A^{pub})$
			\item $B -> A: E_P(E_{K_A^{pub}}(KS))$
			\item $A -> B: E_{KS}(R_A)$
			\item $B -> A: E_{KS}(R_A || R_B)$	
			\item $A -> B: E_{KS}(R_B)$		   
		\end{enumerate}
	
		\bigskip
		
		\textbf{Bilateral Key Exchange with Public Key}\\
		\begin{enumerate}
			\item $B -> A: ID_B || E_{K_A^{pub}}(R_B || ID_B)$
			\item $A -> B: E_{K_B^{pub}}(h(R_B) || R_A || ID_A || KS)$
			\item $B -> A: E_{KS}(h(R_A))$   
		\end{enumerate}
	}
	
	\frame{
	\frametitle{Протоколы распределения ключей}
	\framesubtitle{Смешанные протоколы}
		\textbf{SPX}\\
		$a_A, a_B$ - сетевые адресса.\\
		$L, L_A, L_B$ - время жизни ключей (приватных и публичных) сеанса, пользователя $A$ и пользователя $B$ соответственно.\\
		$m_A = (ID_A || ID_B || L_B || K_B^{pub})$\\
		$m_B = (ID_B || ID_A || L_A || K_A^{pub})$
		\begin{enumerate}
			\item $A -> T: ID_B$
			\item $T -> A: m_A || E_{K_{AT}}(h(m_a)) <- cert_{AB}$
			\item $A -> B: ID_A || E_{K_A^{sec}}(ID_A || K_A^{pub} || L) || $ \\
			$ E_{K_B^{pub}}(KS) || E_{K_A^{sec}}(E_{K_B^{pub}}(KS)) || t || E_{KS}(t) || a_A$   
			\item $B -> T: ID_A$
			\item $T -> B: m_B || E_{K_{BT}}(h(m_B)) <- cert_{BA}$
			\item $B -> A: E_{KS}(t || a_B)$
		\end{enumerate}

	}
	
	\frame{
		\frametitle{Протоколы распределения ключей}
		\subtitle{Протоколы с использованием ЭЦП}
		
		\begin{enumerate}
			\item $A -> B:  E_{K_B^{pub}}(KS || t) || sign_A(ID_B || KS || t)$
			\item $A -> B:  E_{K_B^{pub}}(KS || t || sign_A(ID_B || KS || t))$
			
			\item $A -> B: t || E_{K_B^{pub}}(ID_A ||KS) || sign_A(ID_B || t ||  E_{K_B^{pub}}(ID_A ||KS))$
					   
		\end{enumerate}
		
		\bigskip
		
		\textbf{Сертификаты открытых ключей}
		$cert_A = (ID_A || K_A^{pub} || t || sign_T(ID_A || K_A^{pub} || t))$	
	}
	
	\frame{
		\frametitle{Протоколы распределения ключей}
		\subtitle{Протоколы с использованием ЭЦП}
		
		\textbf{X.509}\\
		$d_A = (T_A || R_A || ID_B || text_1 || E_{K_B^{pub}}(K_A))$ \\
		
		$d_B = (T_B || R_B || ID_A || text_2 || E_{K_A^{pub}}(K_B))$
		\begin{enumerate}
			\item $A -> B: cert_A || d_A || sign_A(d_A)$
			\item $B -> A: cert_B || d_B || sign_B(d_B)$
			
			\item $A -> B: R_B || ID_B || sign_A(R_B || ID_B)$
			\item $A,B:\ \ \ \ \ KS = f(K_A, K_B)$
		\end{enumerate}
	
		Шаг (3) необязателен, выполняется только если нужно подтвержденрие.		
	}

	\frame{
	\frametitle{Протоколы распределения ключей}
	\framesubtitle{Протоколы с использованием ЭЦП}
	
	\textbf{Денниг -Сакко}\\
	\begin{enumerate}
		\item $A -> S: ID_A || ID_B$
		\item $S -> A: cert_A || cert_B$
		\item $A -> B: cert_A || cert_B || E_{K_B^{pub}}(KS || t_A || sign_A(kS || T_A))$
		\item $A,B:\ \ \ \ \ KS = f(K_A, K_B)$
	\end{enumerate}
	
	Шаг (3) необязателен, выполняется только если нужно подтвержденрие.		
	}
	
	\frame{
		\frametitle{Протоколы распределения ключей}
		\framesubtitle{Протокол MTI}
		
		\begin{enumerate}
			\item Предварительный этап: \\
				  Выбираются следующие параметры: $p, \alpha$, где $p$ - простое число, $a \in Z^{*}_p$\\
				  $A$ выбирает $a,\ 1 \leq a \leq p -2$, $z_A = \alpha ^ a\ (mod\ p)$.\\
				  $B$ выбирает $b,\ 1 \leq b \leq p -2$, $z_B = \alpha ^ b\ (mod\ p)$.\\
			\item $A -> B: m_{AB} = \alpha^x\ (mod\ p),\ 1 \leq x \leq p - 2,\ x $ - случайное
			\item $B -> A: m_{BA} = \alpha^y\ (mod\ p),\ 1 \leq y \leq p - 2,\ y $ - случайное
		\end{enumerate}
		
		\bigskip		
		Варианты построения ключа:\\
		\begin{table}[]
			\begin{tabular}{llllll}
				№&  $m_AB$&  $m_BA$&  $K_A$&  $K_B$&  K\\
				1&  $\alpha^x$&  $\alpha^y$&  $m_{BA}^a z^x_B$&  $m_{AB}^b z^y_A$&  $\alpha^{bx+ay}$\\
				2&  $z^x_B$&  $z^y_A$&  $m_{BA}^{a^{-1}}\alpha^x$&  $m_{AB}^{b^{-1}}\alpha^y$&  $\alpha^{x+y}$\\
				3&  $z^x_B$&  $z^y_A$&  $m_{BA}^{a^{-1}x}$&  $m_{AB}^{b^{-1}y}$&  $\alpha^{xy}$\\
				4&  $z^x_B$&  $z^Y_A$&  $m_{BA}^{x}$&  $m_{AB}^{y}$& $\alpha^{bxay}$
			\end{tabular}
		\end{table}
	}
	
	
	\frame{
		\frametitle{Протоколы распределения ключей}
		\subtitle{Предварительное распределение ключей}
		
		Предварительное распределение ключей нужно для уменьшения объёма распределяемой и хранимой информации. \\
		\bigskip
		$A_1, ..., A_n$ - абоненты.\\
		$K$ - множество ключей. \\
		$P$ - множество исходных ключевых параметров ($p_i$ - пароль каждого абонента).\\
		$Q$ - множество значений ключевых материалов абонентов ($q_i$ - секрет каждого абонента). \\
		$R$ - множество значений открытой информации ($r_1, ..., r_n$ - в открытом доступе). \\

		
	}

	\frame{
	\frametitle{Протоколы распределения ключей}
	\subtitle{Предварительное распределение ключей}

	Схема предварительного распределения ключей:\\
	$S(n) = (K,P, Q,R, A_0, A_1)$ \\
	
	\begin{enumerate}
		\item $A_0 : P \times R -> Q$ - алгоритм формирования секретных ключевых материалов.\\
		$A_0(p_i, r_i) = q_i,\ 1 \leq i \leq n$
		\item $A_1 : Q \times R -> K$ - алгоритм вычисления ключа парной связи.\\
		$A_1(q_i, r_j) = A_1(q_j, r_i),$\\
		$K_{ij} = A_1(q_i, r_j), i=j : $ либо не рассматривается либо некий личный секретный ключ.\\
		
	\end{enumerate}
		$A_0(p, r_i) = Q_i \subseteq K^{t_i} \subseteq Q, \ 1 \leq i \leq n$
	}
	
	\frame{
	\frametitle{Протоколы распределения ключей}
	\subtitle{Предварительное распределение ключей}
	
	\textbf{Предложение 1}\\
	$\forall r_i \in R,\ q_i \in Q,\ 1 \leq i \leq n$\\
	$A_0(p, r_i) = q_i $ - имеет одинаковое число решений относительно $p \in P$.
	
	\bigskip
	
	\textbf{Предложение 2}\\
	$\forall r_i \in R,\ k \in K,\ 1 \leq i \leq n$\\
	$A_1(q_i, r_i) = k $ - имеет одинаковое число решений относительно $q_i \in Q$.
	
}	

	
	
	\frame {
		\frametitle{}
		
	}
	
	
	\frame {
		\begin{figure}
			\includegraphics[width=0.8\linewidth]{index1.jpeg}
			
		\end{figure}
	}
	
\end{document}