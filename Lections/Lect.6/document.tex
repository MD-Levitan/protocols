\documentclass{beamer}




\usepackage[T1,T2A]{fontenc}
\usepackage[utf8]{inputenc}

\usepackage{graphicx}
\usepackage{blindtext}
\usepackage{ mathrsfs }
\usepackage[russian,english]{babel}
%\usepackage{ dsfont }

\author{Деркач Максим Юрьевич}
\title{Криптографические протоколы}
\subtitle{Лекция 6 \\ Протоколы на основе техники доказательства знания}


%\usetheme{lucid}
\begin{document}
	\frame {
		\titlepage
	}
	\frame {
		\frametitle{Ссылки}
		
	}
	
	\frame{
		\frametitle{Протоколы на основе техники доказательства знания}
		\subtitle{Определения}
		\textbf{Определение 1}
		Доказательство знания - это интерактивное доказательство, в котором доказывающий
		утверждает проверяющего в том, что он владеет секретной информацией.
		
		\bigskip
		
		Свойства протокола:	
		\begin{itemize}
			\item полнота
			\item корректность
			\item нулевое разглашение	
		\end{itemize}
	}

		\frame{
		\frametitle{Протоколы на основе техники доказательства знания}
		\subtitle{Определения}
		\textbf{Определение 2}
		
		Полнота - свойство означающее, что при выполнении честными участниками 
		протокол решает задачу, для которой создан.
		
		\bigskip
		
		\textbf{Определение 3}
		
		Корректность - свойство протокола противостоять угрозам со стороны злоумышленика,
		не располагающего секретной информацией, но пытающегося выполнить протокол вместо участника $A$,
		которой этой информацией владеет.
		
		\bigskip
		
		\textbf{Определение 4}
		
		Нулевое разглашение - свойство протокола обеспечивающие, что никакая информация о доказываемом 
		утверждение, не может быть получена нечестным проверяющим за полиномиальное время от длины
		переданных сообщений.
		
		
	}

	\frame{
	\frametitle{Протоколы на основе техники доказательства знания}
	\subtitle{Схема протокола}		
		\begin{enumerate}
			\item $A -> B: \gamma$ (заявка - witness)
			\item $B -> A: x$ 	   (запрос - challenge)
			\item $A -> B: y$	   (ответ - response)
		\end{enumerate}

		\bigskip
		
		\begin{enumerate}
			\item $A$ : владеет секретом $S$.
			
			\item $A$ : генерирует  случайное число $r$.
			
			\item $A$ : $\gamma = h(r, S)$ 		
			
		\end{enumerate}
	
		Шаг 3 могут повторять до тех пор пока $B$ не примет решение, что протокол пройден.
		
		Выполнение данного протокола $t$ раз гарантирует, что $A$ - подлинный участник.

	}
	
	\frame{
		\frametitle{Протоколы на основе техники доказательства знания}
		\framesubtitle{Протокол Фиата-Шамира}
				
		$p,\ q$ - простые, $p \neq q,\ n = pq,\ |p|,|q| >= 512 $ \\
		$A$ :  секретный ключ - $a \in  Z_n^*,\ (a, n) = 1$ \\
		\quad \quad открытый ключ  - $b = (a ^ {-1}) ^ 2\ (mod\ n)$ \\
		\begin{enumerate}
			\item $A -> B: \gamma = r ^ 2\ (mod\ n), 1 \leq r \leq n - 1,\ r - $ простое число
			\item $B -> A: x \in \{0,1\}$
			\item $A -> B: y = ra^x \ (mod\ n)$
		\end{enumerate}
	
		 $B$ проверяет $y^2 b ^x \equiv \gamma (mod\ n)$ \\
		 Шаги повторяются $t$ раз. \\

	}

		\frame{
		\frametitle{Протоколы на основе техники доказательства знания}
		\framesubtitle{Протокол Фиата-Шамира}

		\begin{itemize}
			\item Полнота: \\
			$y ^ 2 b ^ x = r ^ 2 a ^{2x} a ^ {-2x} = r ^2 \equiv \gamma (mod\ n)$
			\item Корректность: \\
			\begin{enumerate}
				\item
				$ \forall z, \gamma = z ^2\ (mod\ n) $ \\
				$x = 0 -> y = z$ : вероятность успеха 1\\
				$x = 1 -> y = za,\ a = (\sqrt{b}) ^ {-1}$ : вероятность успеха  $\frac{1}{n}$\\
				Вероятность успешного угадывания: \\
				$P = \frac{1}{2} (1 + \frac{1}{n})$	
				
				\item Выберем $r$ не случайно \\
				$ r = za^{-1},\ \gamma = z ^2 a^{-2} = z^2 b\ (mod\ n) $ \\
				$x = 1 -> y = z$ : вероятность успеха 1\\
				$x = 0 -> y = za^{-1}$ : вероятность успеха  $\frac{1}{n}$\\
				Вероятность успешного угадывания: \\
				$P = \frac{1}{2} (1 + \frac{1}{n})$	
				
			\end{enumerate}	
			$P = \frac{1}{2} (1 + \frac{1}{n}) \approx \frac{1}{2}$
			и $t$ повторов $\rightarrow$ $P_{total} = 2 ^ {-t}$ 
			
			\item Нулевое разглашение: $(y,x,\gamma)$
		\end{itemize}
		
	}

		\frame{
		\frametitle{Протоколы на основе техники доказательства знания}
		\framesubtitle{Протокол Файге-Фиата-Шамира}
		
		$p,\ q$ - простые, $p \neq q,\ n = pq,\ |p|,|q| >= 512 $ \\
		$A$ : \\
		
		секретный ключ - $a = (s_1, ..., s_k),\ s_i \in \  Z_n^*,\ (s_i, n) = 1,\ \forall i \in \{1,..., k\} $ \\
		
		открытый ключ  - $b = (v_1,..., v_k),\  v_i = (s_i ^ 2)^{-1}\ (mod\ n),\ \forall i \in \{1,..., k\} $ \\
		
		\begin{enumerate}
			\item $A -> B: \gamma = r ^ 2\ (mod\ n), 1 \leq r \leq n - 1,\ r - $ простое число
			\item $B -> A: x = (x_1, ..., x_i)\in \{0,1\}^k$
			\item $A -> B: y = r(s_1^{x_1}...s_k^{x_k}) \ (mod\ n)$
		\end{enumerate}
		
		$B$ проверяет $y^2(v_1^{x_1}...v_k^{x_k}) \equiv \gamma (mod\ n)$ \\
		Шаги повторяются $t$ раз. \\
		
	}
	
		\frame{
		\frametitle{Протоколы на основе техники доказательства знания}
		\framesubtitle{Протокол Шнора}

		
		$p,\ q$ - простые, $q | (p -1),\ \alpha \in Z_p,\ ord(\alpha) = q$ \\
		$A$ :  секретный ключ - $ 1 \leq a  \leq  q -2$ \\
		\quad \quad открытый ключ  - $b = \alpha ^ {-a} \ (mod\ p)$ \\
		\begin{enumerate}
			\item $A -> B: \gamma = \alpha ^ r \ (mod\ p),\ 1 \leq r \leq q - 2$
			\item $B -> A: 0 \leq x \leq q - 1$
			\item $A -> B: y = (r + ax) \ (mod\ q)$
		\end{enumerate}
		
		$B$ проверяет $\alpha^y b ^ x \equiv \gamma (mod\ p)$ \\
		Шаги повторяются $t$ раз. \\
		
	}
	
	\frame{
		\frametitle{Протоколы на основе техники доказательства знания}
		\framesubtitle{Протокол Шнора}

		\begin{itemize}
			\item Полнота: \\
			$\alpha^y b ^ x  = \alpha ^{r + ax} \alpha ^{-ax} = \alpha ^{r} \equiv \gamma (mod\ p)$
			\item Корректность: (для $x \in \{0,1\}$ ) \\
			\begin{enumerate}
				\item
				$ \forall z, \gamma = \alpha ^ z\ (mod\ p) $ \\
				$x = 0 -> y = z$ : вероятность успеха 1\\
				$x = 1 -> y = z + a,\ a = - log_{\alpha}b$ : вероятность успеха  $\frac{1}{q}$\\
				Вероятность успешного угадывания: \\
				$P = \frac{1}{2} (1 + \frac{1}{q})$	
				
				\item Выберем $r$ не случайно \\
				$ r = z - a,\ \gamma = \alpha ^ {z - a} = \alpha^z b\ (mod\ p) $ \\
				$x = 1 -> y = z$ : вероятность успеха 1\\
				$x = 0 -> y = z - a$ : вероятность успеха  $\frac{1}{q}$\\
				Вероятность успешного угадывания: \\
				$P = \frac{1}{2} (1 + \frac{1}{q})$	
				
			\end{enumerate}	
			$P = \frac{1}{2} (1 + \frac{1}{q}) \approx \frac{1}{2}$
			и $t$ повторов $\rightarrow$ $P_{total} = 2 ^ {-t}$ 
			
			\item Нулевое разглашение: $(y,x,\gamma)$
		\end{itemize}
		
	}
	
	
	\frame{
		\frametitle{Протоколы на основе техники доказательства знания}
		\framesubtitle{Протокол Окамото}
		
		$p,\ q$ - простые, $q | (p -1),\ \alpha_1,  \alpha_2 \in Z_p $\\
		$ord(\alpha_1) = ord(\alpha_2) = q$ \\
		
		$A$ :  секретный ключ - $(a_1, a_2), \ 1 \leq a_1, a_2  \leq  q -2$ \\
		\quad \quad открытый ключ  - $b = \alpha_1 ^ {-a_1}  \alpha_2 ^ {-a_2}\ (mod\ p)$ \\
		\begin{enumerate}
			\item $A -> B: \gamma = \alpha_1 ^ {r_1} \alpha_2 ^ {r_2} \ (mod\ p),\ 1 \leq r_1, r_2 \leq q - 2$
			\item $B -> A: 0 \leq x \leq q - 1,\ x < 2^t$
			\item $A -> B:			 y_1 = (r_1 + a_1x) \ (mod\ q) $\\
			\quad \quad\quad \quad\ $y_2 = (r_2 + a_2x) \ (mod\ q) $
		\end{enumerate}
		
		$B$ проверяет $\alpha_1^{y_1} \alpha_2^{y_2} b ^ x \equiv \gamma (mod\ p)$ \\
		\begin{itemize}
			\item Полнота: \\
			 $\alpha_1^{y_1} \alpha_2^{y_2} b ^ x  = \alpha_1^{r_1 + a_1x} \alpha_2^{r_2 + a_2x} \alpha_1 ^ {-a_1}  \alpha_2 ^ {-a_2}  = \alpha_1 ^ {r_1} \alpha_2 ^ {r_2} \equiv \gamma (mod\ p)$
			\item Нулевое разглашение: $(y_1, y_2,x,\gamma)$
		\end{itemize}
		
	}

	\frame{
	\frametitle{Протоколы на основе техники доказательства знания}
	\framesubtitle{Протокол GQ}
	
	$p,\ q$ - простые, $n = pq,\ b \geq 3,\ (b,\ \varphi(n))=1 $\\
	
	$A$ :  секретный ключ - $u \in Z_n^*,\ (u,\ n) = 1$ \\
	\quad \quad открытый ключ  - $v = (u^{-1})^b\ (mod\ n)$ \\
	\begin{enumerate}
		\item $A -> B: \gamma = r^b \ (mod\ n),\ 1 \leq r_1, r_2 \leq n - 2$
		\item $B -> A: 0 \leq x \leq b - 1,\ x < 2^t$
		\item $A -> B: y = ru^x\ (mod\ n) $
	\end{enumerate}
	
	$B$ проверяет $v^x y^b \equiv \gamma (mod\ n)$ \\
	\begin{itemize}
		\item Полнота: \\
		$v^x y^b = u^{-bx} r ^b u ^ {bx} = r ^ b \equiv \gamma (mod\ n)$
		\item Нулевое разглашение: $(y,x,\gamma)$
	\end{itemize}
	
	}
	
	\frame{
		\frametitle{Протоколы на основе техники доказательства знания}
		\framesubtitle{Протокол GQ c ключами зависящими от индентификатора}
		
		$p,\ q$ - простые, $n = pq,\ b \geq 3,\ (b,\ \varphi(n))=1,\ ab \equiv 1\ (mod \varphi(n))$\\
		
		$A$ :  секретный ключ - $u = (h(ID_A))^{-a} \in Z_n$ \\
		\quad \quad открытый ключ  - $v = h(ID_A) = (u^{-1})^b\ (mod\ n)$ \\
		\begin{enumerate}
			\item $A -> B: \gamma = r^b \ (mod\ n),\ 1 \leq r_1, r_2 \leq n - 2$
			\item $B -> A: 0 \leq x \leq b - 1,\ x < 2^t$
			\item $A -> B: y = ru^x\ (mod\ n) $
		\end{enumerate}
		
		$B$ проверяет $v^x y^b \equiv \gamma (mod\ n)$ \\
		\begin{itemize}
			\item Полнота: \\
			$v^x y^b = u^{-bx} r ^b u ^ {bx} = r ^ b \equiv \gamma (mod\ n)$
			\item Нулевое разглашение: $(y,x,\gamma)$
		\end{itemize}
		
	}
	
	\frame {
		\frametitle{}
		
	}


	\frame {
	\begin{figure}
		%\includegraphics[width=0.8\linewidth]{index1.jpeg}
		
	\end{figure}
	}

\end{document}