\documentclass{beamer}




\usepackage[T1,T2A]{fontenc}
\usepackage[utf8]{inputenc}

\usepackage{graphicx}
\usepackage{blindtext}
\usepackage{ mathrsfs }
\usepackage[russian,english]{babel}
%\usepackage{ dsfont }

\author{Деркач Максим Юрьевич}
\title{Криптографические протоколы}
\subtitle{Лекция 9 \\ Протоколы распределения ключей (Часть 2). Предварительное распределение ключей}
\setbeamercolor{frametitle}{bg=cyan!10}

%\usetheme{lucid}
\begin{document}
	\frame {
		\titlepage
	}
	\frame {
		\frametitle{Ссылки}
		
		\url{https://habr.com/en/post/431392/}
	}
	
	\frame{
		\frametitle{Предварительное распределение ключей}
		\framesubtitle{Основные понятия и свойства}
		
		Предварительное распределение ключей нужно для уменьшения объёма распределяемой и хранимой информации. \\
		\bigskip
		$A_1, ..., A_n$ - абоненты.\\
		
		$K$ - множество ключей. \\
		
		$P$ - множество исходных ключевых параметров ($p_i$ - пароль каждого абонента).\\
		
		$Q$ - множество значений ключевых материалов абонентов ($q_i$ - секрет каждого абонента). \\
		
		$R$ - множество значений открытой информации ($r_1, ..., r_n$ - в открытом доступе). \\
		
	}
	
	\frame{
		\frametitle{Предварительное распределение ключей}
		\framesubtitle{Основные понятия и свойства}
		
		Схема предварительного распределения ключей:\\
		$S(n) = (K,P, Q,R, A_0, A_1)$ \\
		
		\begin{enumerate}
			\item $A_0 : P \times R -> Q$ - алгоритм формирования секретных ключевых материалов.\\
			$A_0(p_i, r_i) = q_i,\ 1 \leq i \leq n$
			\item $A_1 : Q \times R -> K$ - алгоритм вычисления ключа парной связи.\\
			$A_1(q_i, r_j) = A_1(q_j, r_i),$\\
			$K_{ij} = A_1(q_i, r_j), i=j : $ либо не рассматривается либо некий личный секретный ключ.\\
			
		\end{enumerate}
		$A_0(p, r_i) = Q_i \subseteq K^{t_i} \subseteq Q, \ 1 \leq i \leq n$
	}
	
	\frame{
		\frametitle{Предварительное распределение ключей}
		\framesubtitle{Основные понятия и свойства}
		
		\textbf{Предложение 1}\\
		$\forall r_i \in R,\ q_i \in Q,\ 1 \leq i \leq n$\\
		$A_0(p, r_i) = q_i $ - имеет одинаковое число решений относительно $p \in P$.
		
		\bigskip
		
		\textbf{Предложение 2}\\
		$\forall r_i \in R,\ k \in K,\ 1 \leq i \leq n$\\
		$A_1(q_i, r_i) = k $ - имеет одинаковое число решений относительно $q_i \in Q$.		
	}
	
	\frame{
		\frametitle{Предварительное распределение ключей}
		\framesubtitle{Схема разделения секрета Шамира}
		\textbf{Схема Шамира}\\
		Схема разделения секрета, широко используемая в криптографии. \\
		
		Схема Шамира позволяет реализовать $( k , n )$ — пороговое разделение секретного сообщения (секрета) между $n$ сторонами так, чтобы только любые $k$ и более сторон ($k \leq n$)могли восстановить секрет. При этом любые $k - 1$  и менее сторон не смогут восстановить секрет. 
		
	}

		\frame{
		\frametitle{Предварительное распределение ключей}
		\framesubtitle{Схема разделения секрета Шамира}
		\bigskip
 		
 		\textbf{Первая фаза:} \\
 		\bigskip
		$M$ - секрет, $S(k,n)$ - пороговая схема разделения секрета.\\
		\bigskip
		$p$ - простое , $p > M$, $p$ - известно всем участникам протокола  \\
		\bigskip
		$F(x) = (a_{k-1}x^{k-1} + ... + a_1x +M){\ }mod{\ }p$ - многочлен над полем $Z_p$ \\
		\bigskip
	}

	\frame{
	\frametitle{Предварительное распределение ключей}
	\framesubtitle{Схема разделения секрета Шамира}
	
	\textbf{Генерация секрета:} \\
	
	\bigskip
	$q_i = A_0^{i} = F(i)$ - генерация долей серкрета\\
	
	\bigskip
	Аргументы многочлена (номера секретов) не обязательно должны идти по порядку, главное — чтобы все они были различны по модулю $p$.
	
	\bigskip
	После этого каждой стороне, участвующей в разделении секрета, выдаётся доля секрета — $q_i$ вместе с номером $i$. 
	
	\bigskip
	Помимо этого, всем сторонам сообщается степень многочлена $k - 1$ размер поля $p$.
	
	\bigskip
	Случайные коэффициенты $a_{k - 1}, \dots , a_1 $ и сам секрет $M$ удаляються. 
}


	\frame{
		\frametitle{Предварительное распределение ключей}
		\framesubtitle{Схема разделения секрета Шамира}
		\textbf{Восстановление секрета:}
		\bigskip
		
		Теперь любые $k$ участников, зная координаты $k$ различных точек многочлена, смогут восстановить многочлен и все его коэффициенты, включая последний из них — разделяемый секрет.
		\bigskip
		
		$F(x) = \sum_{i} L_{i}(x)y_i{\ }mod{\ }p$, \\
		
		$L_{i}(x) = \prod_{i \neq j} \frac{x - x_j}{x_i - x_j}{\ }mod{\ }p$,\\
		
		где $(x_i, y_i) \equiv (i, q_i)$ - координаты точек многочлена. 		

	}

	
	
	
	\frame{
		\frametitle{Предварительное распределение ключей}
		\framesubtitle{Схема разделения секрета Блома}
		
		\textbf{Инициализация:}
		\bigskip
		
		$F$ - конечное поле, имеющее достаточно большое число элементов($n$ элементов).\\
		
		\bigskip
		Доверенная сторона (центр распределния) выбирает следующие секретные материалы:
		$1 \leq m \leq n - 2,\ a_{st}$ - секретные материалы, хранимые только в центре распределения. \\
		
		\bigskip
		И строит на их основе полином:
		
		$f(x,y) = \sum_{s=0}^{m} \sum_{t=0}^{m} a_{st}x^sy^t,\ a_{st}=a_{ts},\ s \ne t, \ s,t = 0, ..., m$ \\
				
	}

	\frame{
	\frametitle{Предварительное распределение ключей}
	\framesubtitle{Схема разделения секрета Блома}
	
	\textbf{Добавление участника:}
	\bigskip
	
	Когда новый участник хочет присоединиться к группе, доверенная сторона выбирает для него новый открытый ключ $r_i$. Далее доверенная сторона вычисляет закрытый ключ $q_i$: \\
\bigskip	
		$q_i = (a_0^{(i)}, a_1^{(i)}, ..., a_m^{(i)})$\\
	
		$q_i(x) = f(x, r_i) = a_0^{(i)} + a_1^{(i)}x + ... + a_m^{(i)}x^m$\\
	\bigskip
	
	Открытый и закрытый ключ сообщаются участнику по надёжному каналу без прослушивания. 

	
}
	
		\frame{
		\frametitle{Предварительное распределение ключей}
		\framesubtitle{Схема разделения секрета Блома}
		
		\textbf{Установление сессии:}
		\bigskip
	
		
		$K_{ij} = K_{ji} = f(r_i, r_j) = q_i(r_j) = q_j(r_i)$ \\
	\bigskip
		$A_i$ хранит $m+1$ значение ключевых паролей. \\
		Схема Блома является стойкой к  $m$-кратной компрометации ключей.
		
	}
	
	\frame {
		\begin{figure}
			%\includegraphics[width=0.8\linewidth]{index1.jpeg}
			
		\end{figure}
	}
	
\end{document}