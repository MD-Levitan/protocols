\documentclass{beamer}




\usepackage[T1,T2A]{fontenc}
\usepackage[utf8]{inputenc}

\usepackage{graphicx}
\usepackage{blindtext}
\usepackage{ mathrsfs }
\usepackage[russian,english]{babel}
%\usepackage{ dsfont }

\author{Деркач Максим Юрьевич}
\title{Криптографические протоколы}
\subtitle{Лекция 10 \\ Протоколы голосования}


%\usetheme{lucid}
\begin{document}
	\frame {
		\titlepage
	}
	\frame {
		\frametitle{Ссылки}
		
		%\begin{enumerate}
 
		%\end{enumerate}
	}
	
	\frame{
		\frametitle{Протоколы голосования}
		\framesubtitle{Понятие протокола голосования}
		
		Задача голосования заключается в том, что несколько абонентов должны 
		совместно выбрать решение из некоторого множества возможных решений.
		Каждый абонент заполняет свой бюллетень, отражающий решение этого агента. 
		Совместное решение вырабатывается путем обработки всех бюллетеней.
		\\
		
		Часто при процедуре голосования важно обеспечить конфеденциальность решений,
		принимаемых абонентами. Данная задача рещается при помощи специальных протоколов.
		
		
	}

	\frame{
		\frametitle{Протоколы голосования}
		\framesubtitle{Понятие протокола голосования}
		
		Наиболее часто используются такие протоколы голосования, в которых
		\begin{enumerate}
			\item каждый агент отсылает свой бюллетень некоторому доверенному агенту, 
				  называемому \textbf{Центральной Избирательной Комиссией(ЦИК)}, которого 
				  мы будем обозначать ниже символом $s$;
			\item ЦИК обрабатывает полученные от агентов бюллетени, и публикует результаты голосования.
		\end{enumerate}

	}

		\frame {
		\begin{figure}
			\includegraphics[width=1\linewidth]{cik.jpeg}
			
		\end{figure}
	}

	\frame{
		\frametitle{Протоколы голосования}
		\framesubtitle{Понятие протокола голосования}
		
		Условия на процедуру голосования, которые должен обеспечить протокол, могут имет, например, следующий вид.
		\begin{enumerate}
			\item Голосовать могут только те агенты, которые имеют на это право.
			\item Каждый избиратель может голосовать только один раз.
			\item Невозможно установить, за кого проголосовал каждый избиратель.
			\item Невозможно использовать дубликат заполненного бюллетеня.
			\item Невозможно изменить результат голосования каждого избирателя.
			\item Каждый избиратель может проверить, что его бюллетень учтён.
			\item Всем известно, кто участвовал в голосовании.
		\end{enumerate}
		
		
	}

	\frame{
		\frametitle{Протоколы голосования}
		\framesubtitle{Примеры}
		
		\textbf{Протокол 1}\\
		$u_i$ - бюллетень избирателя $a_i$\\
		\begin{enumerate}
			\item $A_i -> S: E_{K_{A_iS}}(u_i)$
			\item $S$ вычисляет результат и публикует его.
		\end{enumerate}
		В данном протоколе не выполняются почти все вышеперечисленные условия.
		
		\bigskip
		
		\textbf{Протокол 2}\\
		\begin{enumerate}
			\item $A_i -> S: E_{K_{A_iS}}(sign_{A_i}(u_i))$
			\item $S$ вычисляет результат и публикует его.
		\end{enumerate}	
		
	}
	
	\frame{
		\frametitle{Протоколы голосования}
		\framesubtitle{Протоколы голосования с ЦИК и ЦУР}
		
		Для противодействия возможной нечестности со стороны ЦИК, можно использовать
		\textbf{Центральное Управление Регистрации (ЦУР)}. Необходимое условие корректности 
		протокола является отсутствие обмена между ЦУР и ЦИК.
		\begin{enumerate}
			\item $A_i -> S^{'}: request$
			\item $S^{'} -> A_i: R_i$ - (случайный регистрационный номер).
			\item $S^{'} -> S: \Re$ - список всех выданных регистрационных номеров.
			\item $A_i -> S: (ID_i, r_i, u_i)$
			\item $s$ проверяет: $r_i \in \Re ?$. Если верно, то
				\begin{enumerate}
					\item $\Re = \Re \backslash \{ r_i\}$
					\item $\Im = \Im  \cup \{ID_i\}$ (в начале работы $\Im = \emptyset$) 
				\end{enumerate}	
			\item после получения всех бюллетеней $s$ публикует результат, и список записей вида $(ID_i, u_i)$
			\item $S^{'}$ публикует всех список зарегистрированных $a_i$
		\end{enumerate}		
		\bigskip
	}
	
	\frame{
		\frametitle{Протоколы голосования}
		\framesubtitle{Улучшенный протокол голосования}
		
		\begin{enumerate}
			\item $S$ публикует список всех абонентов, имеющих право голосовать
			\item $A_i -> S: intention$
			\item $S$ публикует список всех избирателей, собирающихся принять участие в выборах
			
			\item $S -> A_i: ID_i$ - (регистрационный номер).
			\item $A_i -> S: ID_i, E_{K^{pub}_{A_i}}(ID_i, u_i)$
			\item $S$ публикует $E_{K^{pub}_{A_i}}(ID_i, u_i)$.			
			\item $A_i - > S: ID_i, K^{sec}_{A_i}$
			\item $S$ расшифровывает бюллетени и обрабатывает их
			\item $S$ публикует результаты голосования, и все $u_i, E_{K^{pub}_{A_i}}(ID_i, u_i)$
			\item Если $A_i$ обнаружил, что его $u_i$ учтен неверно, то  \\
					$A_i -> S: ID_i, E_{K^{pub}_{A_i}}(ID_i, u_i), K^{sec}_{A_i}$
			\item Если $A_i$ хочет изменить свой выбор, то  \\ 
					$A_i -> S: ID_i, E_{K^{pub}_{A_i}}(ID_i, u_i^{'}), K^{sec}_{A_i}$
			
		\end{enumerate}		
		\bigskip
	}


	\frame{
		\frametitle{Протоколы голосования}
		\framesubtitle{Выбор без ЦИК}
		
		Рассмотрим случай, когда в голосовании участвуют 4 избирателя, которые используют одну и ту же
		асимметричную КС.
		$m_i = E_{K_{A_1}^{pub}}(u_{i1}, r_{i1})$\\
		$u_{i1} = E_{K_{A_2}^{pub}}(u_{i2}, r_{i2})$\\
		$u_{i2} = E_{K_{A_3}^{pub}}(u_{i3}, r_{i3})$\\
		$u_{i3} = E_{K_{A_4}^{pub}}(u_{i4}, r_{i4})$\\
		$u_{i4} = E_{K_{A_1}^{pub}}(v_{i1})$\\
		$v_{i1} = E_{K_{A_2}^{pub}}(v_{i2})$\\
		$v_{i2} = E_{K_{A_3}^{pub}}(v_{i3})$\\
		$v_{i3} = E_{K_{A_4}^{pub}}(v_{i, r_{i5}})$\\
		$v_i$ - бюллетень избирателя $A_i$
	}

		\frame{
		\frametitle{Протоколы голосования}
		\framesubtitle{Выбор без ЦИК}

		
		
		\begin{enumerate}
			\item $\forall i \ A_i -> A_i: m_i$ 
			\item $A_1 -> A_2: \{u_{i1\ |\ i = 1, ..., 4}\}$
			\item $A_2 -> A_3: \{u_{i2\ |\ i = 1, ..., 4}\}$
			\item $A_3 -> A_4: \{u_{i3\ |\ i = 1, ..., 4}\}$
			\item $A_4 -> A_1: \{u_{i4\ |\ i = 1, ..., 4}\}$
			\item $A_1 -> \{A_2, A_3, A_4\}: \{sign_{A_1}(v_{i1})\ |\ i=1,...,4 \}$
			\item $A_2 -> \{A_1, A_3, A_4\}: \{sign_{A_2}(v_{i2})\ |\ i=1,...,4 \}$
			\item $A_3 -> \{A_2, A_1, A_4\}: \{sign_{A_3}(v_{i3})\ |\ i=1,...,4 \}$
			\item $A_4 -> \{A_2, A_3, A_1\}: \{sign_{A_4}(v_{i4})\ |\ i=1,...,4 \}$
		\end{enumerate}		
		\bigskip
	}

	
	\frame {
		\frametitle{}
		
	}
	
	
	\frame {
		\begin{figure}
			\includegraphics[width=0.8\linewidth]{index1.jpeg}
			
		\end{figure}
	}
	
\end{document}